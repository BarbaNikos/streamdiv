\section{Problem Definition}
\subsection{Input Output Relationship}
	Let $k$ be the number of tuples of $S$, where $S = \{t_0, t_1, \dots, t_{k-1}\}$ representing the top-$k$ set.  Let $n$ be the number of tuples of $N$, where $N = \{t'_0, t'_1, \dots, t'_{n-1}\}$ representing the set of incoming tuples from the stream.  (Note for the Incremental Replacement Scheme, $n = 1$.)  Our goal is to take some tuples from $N$ to replace some tuples in $S$ based on recency, relevancy, and diversity.

\subsection{Recency}
	Recency is defined as a monotonically decreasing score, computed as some formula relating the tuple's timestamp of arrival to the system $T_A$ and the current system time $T_C$.  We choose two different formulas to capture a stronger and weaker emphasis on the recency score of a tuple $Rel(t)$.
	\begin{enumerate}
		\item $Rec(t) = -|T_C - T_A|$
		\item $Rec(t) = e^{-|T_C - T_A|}$
	\end{enumerate}
	The first provides a linear decay, allowing older tuples the possibility to persist longer in the top-$k$ set.  The second provides an exponential decay, which heavily prefers newer tuples.

\subsection{Relevancy}
	Relevancy $Rel(t)$ is a user-defined function regarding what tuples are considered relevant in a user's query.  We provide an explanation as to what constitutes Relevancy in Section IV, as the definition is dependent on the query.

\subsection{Diversity}
	Diversity is a complex operation divided into three parts: Replacement Policy, Distance Definition, and Distance Threshold.  

As far as different dimensions of the problem are concerned, ranking is
considered as a set of user-defined properties that are either attached to
incoming tuples (annotate), or are used for filtering out tuples. Turning
to the diversity operator, we abstract it as a stateful operator, with time-
based sliding windows. The operator's frequency of result production
can be one of the following: (i) a temporary snapshot of the most diverse
tuples in the current time-window produced every time a tuple is received,
(ii) a set of the most diverse tuples among all the tuples that have been
received in a user-defined epoch. In the latter case, we consider how
many tuples are replaced and which ones every time a new tuple arrives.
For the former case, we consider a static approach (a pre-defined fraction
p of the tuples are replaced after every epoch), and a dynamic one (the
fraction p of the tuples replaced is proportional to the $\delta$ in diversity gained
compared to the replacement fraction of the previous epoch). As regards
to replacement policy, we have the alternatives of random replacement,
and least recently tuple received replacement.