\section{Introduction}

Data creation and collection only continues to increase, leaving a large burden on analysts to come up with a way to quickly and efficiently comb through the data to locate trends and correlations.  There tends to be a focus on data-specific approaches with frameworks that work well for certain kinds of data (but not well on others CITATION NEEDED).  There also is a need in data analysis to provide relevant but diverse data points as a digest of the full data set.  Several offline versions have been created \cite{Ge:2015:PD:2795218.2795224} CITATIONS NEEDED, but lack the speed to work in a streaming environment.  We present StreamDiv, a streaming framework built on top of a previously developed diversity system for offline database systems, PrefDiv\cite{Ge:2015:PD:2795218.2795224}, to provide queriers of a data streaming systems a top-$k$ set that is most relevant and diverse, adding recency as a possible influence on the digested set.

The system
model of our work is a stand-alone stream processing system, with a single
stream of data as input. Processing is modeled as a pipeline of operators,
which process incoming tuples and forward results to the next operator.
The output is not a single result (as in the case of traditional databases),
but periodical results that reflect the most sensible outcome (according
to the algorithm) based on time-windows.\\

\indent As far as different dimensions of the problem are concerned, ranking is
considered as a set of user-defined properties that are either attached to
incoming tuples (annotate), or are used for filtering out tuples. Turning
to the diversity operator, we abstract it as a stateful operator, with time-
based sliding windows. The operator's frequency of result production
can be one of the following: (i) a temporary snapshot of the most diverse
tuples in the current time-window produced every time a tuple is received,
(ii) a set of the most diverse tuples among all the tuples that have been
received in a user-defined epoch. In the latter case, we consider how
many tuples are replaced and which ones every time a new tuple arrives.
For the former case, we consider a static approach (a pre-defined fraction
p of the tuples are replaced after every epoch), and a dynamic one (the
fraction p of the tuples replaced is proportional to the $\delta$ in diversity gained
compared to the replacement fraction of the previous epoch). As regards
to replacement policy, we have the alternatives of random replacement,
and least recently tuple received replacement.